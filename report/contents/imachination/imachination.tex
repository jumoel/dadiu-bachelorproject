



\section{Imachination}
                + Beskriv spillet og screendumps
Spillet skal forestille at være produktet af et barns fantasi. Drengen leger i sin sandkasse at han graver sig til kina og for at komme igennem jorden må han bygge en robot og kæmpe sig vej mod underlige væsener og andre robotter.
Spillet er todelt i en bygge del, hvor man skal konstruere sin robot fra forskellige våben, og en "RTS" del hvor spilleren skal manurere sin robot rundt igennem underjordiske grotter hvor der skal bekæmpes modstandere.

[Screenshot af garage]
Man bygger sin robot ved at drag-n-droppe våben ned på ens robot. Det er ligeledes muligt at fjerne eller omrokere våben.

[Screenshot af RTS]
Man kører ved at holde musen nede og flytte cursoren i den retning man ønsker at bevæge sig. Robotten roterer hele tiden for at holde fronten mod cursoren. Det er muligt at trykke på sig selv og derved affyre sin repulsor. Målet er at positionere sigselv så det er muligt at ødelægge de 'hives' der befinder sig rundt i banerne og derefter at besejre bossen uden at dø.

\subsection{Forløb}
                + Roller
kodeslave
                + generelle forløb (scrum, arbejdsmetoder)

scrum møder hver morgen kl 9
2delt prototyping

                + fokus (hvad har VI kodet)

movement
prototyping af våben
merge de 2 verdener
impl. AI, grid, pathfinding, stlålet fra nettet
animationer på AI
animationer og UI på spilleren


                + hvad er vi blevet nødt til at lave

hjælpe alle artister med at få deres lort ind.

                + problemer og udfordringer
movement blev lavet om 3 gange.
simplemove -> fysik

          o AI i vores spil og hvordan de relaterer sig til vores opgave
                + hvad er der af AI i vores spil

pathfinding melee AI
pathfinding boss med ranged weapons

                + hvilken rolle spiller AI’en i spillet
melee AI er bare kanonføde så der er lidt underholdning mens man kører rundt i level.
boss ai er forhindringen til næste level.
