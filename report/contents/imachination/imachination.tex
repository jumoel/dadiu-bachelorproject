\section{Imachination}
Spillet viser produktet af et barns fantasi. En dreng leger i sin sandkasse, at han graver sig til kina, og for at komme igennem jorden må han bygge en maskine og kæmpe sig vej mod underlige væsener og andre maskiner.
[Screenshot af progress gennem jorden]
Der er taget udgangspunkt i børns betagelse af store maskiner. Især drenge er meget fascineret af hvordan maskiner virker, og det er som regel altid et hit med køretøjer eller maskiner blandt drengebørn. Vi har i spillet prøvet at opnå en ``wauw'' effekt ved at se på hvordan børn fantaserer og leger. ``Hvad er endnu bedre end en kanon, en større kanon''.
Spillet er todelt i en bygge del, hvor man skal konstruere sin maskine fra forskellige våben, og en "RTS" del hvor spilleren skal manurere sin maskine rundt igennem underjordiske grotter hvor der skal bekæmpes modstandere.

[Screenshot af garage]
Man bygger sin maskine ved at drag-n-droppe våben ned på ens maskine. Det er ligeledes muligt at fjerne eller omrokere våben.

[Screenshot af RTS]
Man kører ved at holde musen nede og flytte cursoren i den retning man ønsker at bevæge sig. maskinen roterer hele tiden for at holde fronten mod cursoren. Det er muligt at trykke på sig selv og derved affyre sin repulsor. Målet er at positionere sigselv så det er muligt at ødelægge de 'hives' der befinder sig rundt i banerne og derefter at besejre bossen uden at dø.


\subsection{Forløb}

Før selve produktionen startede var vi en håndfuld fra gruppen der satte os ned og snakkede om hvilken type spil vi kunne lave og hvilket tema. Vi var en gruppe bestående af vores game director, game designer, project manager, art director, visual artist og en programmør. vi fik sorteret i nogle tidlige brainstorm ideer og fastlagt en mere konkret idé. Den første dag af produktionen blev idéen pitched til hele gruppen. Da den faldt i god jord sparede vi værdifuld tid i preproduktionen da vi allerede få timer efter at have mødtes kunne starte på at producere prototyper.

For programmørenes del startede udviklingen todelt. Da vores spil består af to vidt forskellige dele, bygge delen og RTS delen, og derfor startede vi med at lave prototyper inddelt i 2 grupper. Sammen med en anden programmør blev jeg sat til at prototype en kasse der kunne følge efter musen i et RTS lignende kameraview. De to dele skulle så senere sættes sammen i spillet.

Vores projekt blev kørt med scrum møder hver morgen hvor hver enkelt fortalte hvad der var blevet  lavet dagen før, og hvad der var planlagt at man skulle bruge dagen på.
Derudover var der ikke meget styring af udviklingen. Der var lead-møder hvor leads fra hvert felt snakkede med director, designer og project manager, men udover det var der ikke meget koordinering. Internt på programmør holdet havde vores lead et vist overblik over hvor langt vi var nået, men deudover arbejdede vi bare derudaf.




\subsection{Mit fokus}
Ved første udkast var jeg den programmør der skulle bruge mest tid på AI'en grundet min bacheloropgave. Det var idéen at jeg ville kunne bruge de to midterste af de fire uger på at udvikle AI'en til spillet. Det vidste sig hurtigt at det var et meget urealistisk bud da holdet var meget tidspresset af blot at få spillet færdigt uden advanceret AI. Derfor blev det ikke til så meget AI som jeg havde håbet, men en del mere implementering af diverse ting til spillet.

Som førnævnt blev vores prototype udvikling opdelt i to grupper. Jeg skulle lave vores diablo-lignende bevægelses system, hvor maskinen skulle følge efter musen hvis man holdt museknappen nede. Det blev lavet ved at raycaste mod det plan maskinen kører i og derefter hele tiden at forsøge at rotere maskinen så den vender mod raycast-hittet. Disse raycasts bliver lavet hver update uanset om musen er trykket ned eller ej. Hvis man trykker på musseknappen begynder maskinen at køre i fremadretningen uanset hvilken vej det måtte være. I den første prototype blev maskinen flyttet med translateringer, hvilket ødelagde det fysiksystem Unity har indbygget. Det blev ændret således at maskinen får tilføjet en kraft i dens fremadretning. Det gjorde derimod at bliven beholder en kraft i en retning selvom den drejer hvilket får maskinen til at skride ud. Det er måske meget godt hvis man har en normal bil, men da vores spil indeholder en stor maskine med larvefødder måtte vi ændre alt kraft maskinen får fra accelerering til at være i fremadretningen også selvom denne skulle ændres. 
Vi blev også nødt til at begrænse kraften maskinen kunne få på af y aksen. Det kunne forekomme at bossen og spilleren fik lavet sammenstød der skabte enorm opdrift. Derfor at der blot sat en maks begrænsning på hvor meget opdrift maskinen kan have på et givent tidspunkt. Dette forhindrer maskinen i at flyve højt op ved mindre der konstant bliver tilføjet kraft til den.

Vi havde gentagne gange testet spillet på børn fra det lokale fritidshjem og børnene kunne finde ud af at manurere rundt i spillet, og adspurgt havde de intet at klage over. Alligevel viste det sig at når voksne spillede spillet følte de ikke at maskinen reagerede helt som de ville. Det skyldes nok dels at børn er hurtigere til at acceptere at tingene er som de er, frem for at ville have ting til at opføre sig som de vil have. Derudover skyldes det nok at børnene gerne ville være gode til at hjælpe og gode til at spille spillet, så de gav ikke udtryk for deres problemer på sammen måde. Vi indså derfor at vi måtte gøre så meget som muligt for at hjælpe spilleren til dels at vide hvilken vej han kunne forvente at hans maskine ville køre på et givent tidspunkt, og dels at gøre det nemmere og sjovere at køre.

Da vores styring minder meget om at have en snor bundet fast i maskinens forende valgte vi at prøve at genskabe denne følelse bedre, da det var det brugeren forventede. Vi valgte at sætte drejehastigheden op ud fra hvor stor vinklen var mellem maskinens fremadretning og den ønskede retning. Hvis man trække maskinen 180 grader modsat af hvor fremadretningen er vil maskinen derfor dreje meget hurtigt og derved dreje på et lille område selvom der køres fremad gennem hele svinget, svarende til at man havde en snor bundet fast i en kasses forende og trak den modsatte vej. 

Det viste sig også at være et problem for voksne at se hvilken del af maskinen der var forenden. På tros af at maskinen altid forsøger at rotere mod musen. Derfor implementerede vi to styringsplaner omkring maskinen som indeholder GUI der skal fortælle spilleren hvordan maskinen vender (det orange lag) samt hvilken retning maskinen forsøger at dreje imod (den grønne pil).
[Screenshot af GUI planer]

Jeg lavede også en del prototyping af våbene til maskinen. Det var den oprindelige plan at der skulle laves fem våben til at starte med, og den overskydende til i projektet skulle bruges på at lave flere, sjovere og skørere våben. Det viste sig dog at der ikke var noget overskydende tid så dem fem oprindelige våben blev til de endelige våben. Flammekasteren og saven er lavet med collidere der har en \textbf{on trigger stay} funktion. Så længe en fjende opholder sig inde i collideren tager den skade. Dynamitkasteren og gravearmen bruger begge en sphere collider hhv. fra nedslagspunktet fra dynamitstangen eller skovlen og jo tættere på centrum en anden collider befinder sig jo mere skade og jo mere kraft bliver der tilføjet til fjenden. Dynamitstangen får blot tilføjet en kraft i en retning og exploderer når den rammer en anden collider. Laseren bruger raycasting til at se hvad den rammer.

Det blev besluttet af vores lead programmør at vi skulle implementere en allerede udviklet A\* pathfinding som vi fandt på nettet. Implementeringen havde både A\* pathfinding samt en modul til at generere et grid med et nemt brugerinterface. Desværre var det ikke helt så smertefrit at implementere dette som først antaget. Vi fik nogle performance problemer og da vi i spillets tredje bane har 50 små fjender som jagter dig på en gang måtte dele af koden omskrives for at bruge mindre ressourcer. Det var hovedsageligt selve styrringen / flytningen af enheder der tog mange kræfter. Derfor er løsningens originale måde med Unitys \textbf{charactermove} udskiftet med at bruge fysik til at flytte fjenderne rundt i deres fremadretning. Da denne løsning ikke kunne bruge knudernes højde og fjenderne derved f.eks. ikke går op ad bakker men derimod igennem dem brugte vi Unitys indbyggede funktion til at se hvor højt et terræn er på et givent punkt. Fjenderne translateres derefter op i y-retningen så de kommer til at stå oven på terrænet istedet for nede i jorden eller oppe i luften.

Jeg arbejdede en del sammen med holdets animator da han skulle animere de små fjender. Når først vi havde overvundet de første problemer med at få tingene til at spille sammen med Unity så var det ganske nemt at crossfade mellem animationer og derved få det til at se nogenlunde glidende ud.  

\subsection{Problemer og udfordringer}

Undervejs i projektet gennemgik vores styrringssystem flere itterationer. Vi havde mange problemer med at få maskinen til at opføre sig naturligt uden at introdusere nye problemer. F.eks. ville vi gerne have maskinen til at tippe hvis den holdt på en skrå flade, men den måtte samtidig ikke kunne tippe hele vejen over. Dette løste vi ved at fastlåse den maksimale hældning maskinen kan have.

Vi havde også en del problemer med feedback i vores spil. Undervejs i udviklingen var det en meget tom fornemmelse at køre rundt og skyde ting. Derfor blev der gjort noget ud af at tilføje feedback til både våben og fjender. Laseren har fået tilføjet et rekyl så hvis man sætter flere på den ene siden af maskinen end den anden vil man rykke til den modsatte side når de skyder. Fjenderne flyver ud af billedet med ild i halen når de bliver slået ihjel og de hives hvor de kommer fra skubber en væk når de skyder.

Undervejs i produktionen blev der brugt meget af programmørernes tid på at hjælpe alle de andre med Unity. Det blev programmørernes opgave at indsætte alt visuelt og lyd i spillet fordi ingen tidligere havde fået introduktion til Unity. Det var meget naturligt at det var programmørerne der fik mest og hurtigst kendskab til Unity og derfor gik der desværre rigtig meget tid til spilde ved at hjælpe de andre.

\subsection{AI i spillet}

Vores små fjender bruger en simpel A\* pathfinding algoritme. Der bliver genereret et grid indenfor en boundingbox ved hjælp af raytracing kastet højt oppe fra. Ved alle intersects bliver der lavet en knude. Hvis rayen kolliderer med et object tagged som ``wall'' kan knuden ikke benyttes. Derfor skal alle objecter der ikke kan paseres tagges som ``wall''.
De finder derefter den korteste vej til spilleren og roterer så deres \textbf{forward}-vektor peger mod den næste knude i stien. De bliver hele tiden flyttet frem i deres fremadretningen. Hvis de vender væk fra den vej de gerne vil bliver bevægelseshastigheden sat ned proportionalt med vinklen mellem den ønskede retning og den nuværende retning.
Når en fjende kommer tæt nok på spilleren stopper den bevægelse mod fremadretningen og roterer blot mod spilleren samtidig med den angriber. 
Oprindeligt var der lagt op til to typer fjender. Vi ville have implementeret nogle skydende versioner af de små fjender, sådan at de ikke alle blot løber hen i døden, men på grund af tidsmangel blev disse fraskrevet.

Vores boss AI bruger samme pathfinding som de små fjender. Den har blot fået tilføjet en feature hvor den vurderer om den kan ramme med nogle af dens skydevåben på afstand. Hvis den ser at den kan ramme med et skydevåben (spilleren er indenfor eller i den rigtige afstand, samt der kan raycastes direkte hen og ramme spilleren uden at ramme forhindringer) roterer den således at den får peget et skydevåben i spillerens retning. Jo længere tid der bliver brugt på at skyde med skydevåbene jo højere vægt får disse og bossen bliver derfor mere tilbøjelig til at skifte strategi og køre hen til spilleren og forsøge at bruge nærkampsvåben. Denne vægt falder så over tid så hvis man kører væk fra bossen skifter han taktik til igen at skyde efter en.

I det originale design var det meningen at vi ville have implementeret seek and flee, samt en vægtning af våben så bossen kunne se om han var stærkere i nærkamp end dig og ligeledes på afstand. På grund af tidsmangel i projektet blev disse ting dog skåret fra, for da spillet henvender sig til 6-7 årige børn vurderede holdet at der var lille udbytte i at have disse features frem for andre.

Vores små fjender er dem der skaber liv i banerne. Deres formål er til dels at jagte spilleren så der kommer masser af bevægelse i spillet, dels at blive pløjet ned så spilleren får lov at se sine våben udfolde sig og skyde modstandere. Det er disse fjender som vil skabe grundlaget for afprøvningen af AI'en i spillet da det er disse som skaber mulighederne for både alternativt gameplay samt guider spilleren hen mod de hives der spawner fjenderne og derved leder spilleren mod dens mål. Hvis der havde været mere tid til projektet havde disse fjender været en oplagt mulighed at udvide. Sammen med nye våben kunne de f.eks. dø på forskellige måder alt efter hvilket våben de er blevet skudt med og derved give en masse feedback til spilleren.

Bossen spiller to roller. Den ene rolle er udfordringen som du skal overkomme for at gøre dig fortjent til at spille næste bane. Da de små fjender dør ekstremt let er det en smule mere tilfredsstillende at få lov til at slå en lidt større maskine ihjel. Derudover giver bossen et nyt våben som belønning hvis det lykkedes spilleren at vinde kampen.
Den anden rolle bossen spiller er en form for tips og tricks til spilleren. Da der i spillet ikke er en særlig god tutorial har bossen også til formål at vise et eksempel på hvordan man kunne bygge sin maskine.


