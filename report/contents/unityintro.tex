
\chapter{Introduktion til Unity}
\label{cha:unityintro}

Unity er en \emph{WYSIWYG}\footnote{What You See Is What You
  Get}-editor til at lave spil i.

% [TODO: hvordan forklarerer man hvordan Unity fungerer?]

\section{Assets}
\label{sec:assets}
Alle filer i et Unity projekt er \emph{assets}. De samme \emph{assets}
kan instantieres flere gange i hver bane, eller bruges i flere
forskellige baner. Eksempler p� assets inkluderer lydfiler, teksturer,
3D modeller og \emph{scripts}. Man kan ud lave \emph{prefabs} ud af
\emph{assets} inde i Unity, disse er sammens�tninger af andre
\emph{assets}.

\section{Triggers}
\label{sec:triggers}

En trigger kan teste om et objekt st�der ind i triggerens kollisions
boks. Denne boks kan v�re defineret som et usynligt objekt eller som
noget spilleren kan se og st�de ind i. Triggers kan bruges til at
aktivere funktionalitet n�r fx spilleren g�r et bestemt sted hen. Det
er fx triggers der afg�r om man er kommet igennem en bane i \cat.

\section{Koordinatsystemet}
\label{sec:unitykoordinatsystem}

Hvis vi igennem projektet bruger $X$, $Y$ og $Z$ til koordinater, vil
bev�gelse langs $X$ og $Z$ akserne svare til bev�gelse p� jorden. Det
vil sige at figuren ikke kan flyve eller falde uden bev�gelse langs
$Y$ aksen.



hvordan unity fungerer
triggers
assets
koordinatsystem

%%% Local Variables: 
%%% mode: latex
%%% TeX-master: "../report"
%%% End: 
