
\chapter{Introduktion til Unity}
\label{cha:unityintro}

Unity er en den \emph{editor}, vi brugte p� vores
DADIU-produktionener. Den er baseret p� \emph{WYSIWYG}\footnote{What
  You See Is What You Get}-princippet. Unitys prim�re form�l er at
g�re udviklingsprocessen let.

Alle projekter i Unity indeholder en lang r�kke \emph{assets}:
Lydfiler, grafik, videoer, teksturer, 3D-modeller og
\emph{scripts}. Disse objekter kan man kombinere til nye
'superobjekter', som i Unity g�r under navnet
\emph{prefabs}. Fordelene ved at udnytte dem, er at man kan genbruge
komplicerede objekter p� tv�rs af baner og n�jes med at �ndre eller
opdatere dem �t sted. En bane i et Unity-spil kaldes en
'scene'. Scener kan dog ogs� bruges til \emph{loading}-sk�rme, menuer,
mellemsekvenser og rulletekster. Unity p�f�rer dermed den
objektorienterede tankegang til en \emph{drag and drop}-editor.

Unity l�gger op til, at man programmerer sine spil, s� de er
\emph{event}-baserede. Der er indbygget funktionalitet til
\emph{trigger}-zoner, hvor man automatisk kan s�rge for, at der sker
noget n�r et objekt kommer ind i zonen, opholder sig i zonen eller
forlader zonen. Det kan man for eksempel bruge til at have
omr�despecifik kamerastyring.

Koordinatsystemet i Unity er vendt s�ledes, at $(x,~z)$-planet er
vandret og $y$-aksen peger opad.

%%% Local Variables: 
%%% mode: latex
%%% TeX-master: "../report"
%%% End: 
