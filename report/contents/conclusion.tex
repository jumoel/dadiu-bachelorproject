
\chapter{Konklusion}
\label{cha:konklusion}

Vi har af vores test set at KI'en kan give grobund for alternative
spiloplevelser i form af leg. Vi har dog ogs� erfaret, at KI'en kan
modvirke alternative spilopvelser. I \ima fik b�rnene gl�de af at lege
med den kunstige intelligens, mens vi i \cat s�, at KI'en direkte
modvirkede alternative m�l, da b�rnene fokuserede mere p� ikke at d�,
end at udforske og lege, for eksempel med kakerlakkerne. Vi har dermed
erfaret, at man med KI i h�j grad kan p�virke den m�de spilleren
spiller spillet p�, men at man ogs� har mulighed for at styre
spilleren ud i, eller v�k fra delm�l. Dette s�s tydeligt i \sop, hvor
KI'en aktivt �ndrede de prim�re veje, spilleren tog gennem banen.

I rapporten har vi beskrevet vores udviklingsforl�b hos DADIU og vores
arbejde i Unity. Vi har givet et indblik i de problemer og
udfordringer vi st�dte p� undervejs, og har givet vores bud p�
forbedringer en fremtidig produktion kan drage nytte af.

Vi har udvidet vores spil med heatmap funktionalitet, der g�r os i
stand til at registrere hvor i spillene en spiller bev�ger sig, samt
hvordan tidsfordelingen ligger i forskellige omr�der i
banerne. Derudover kan vi se, hvor lang tid det tager spilleren at
f�rdigg�re baner, samt se hvor de eventuelt d�r. Disse data har vi
anvendt til at skabe et billede af den kunstige intelligens p�virkning
i de forskellige spil.

Vi betragter vores projekt som v�rende succesfuldt, da vi har haft tre
vellykkede praktikforl�b hos DADIU, som alle har resulteret i fine
produkter. Vores heatmapfunktionalitet har givet os en mulighed for at
vurdere den kunstige intelligens p�virkning p� spillerens progression
i vores spil. Desuden har vi kunnet vurdere og diskutere, i hvor h�j
grad den kunstige intelligens spiller en rolle -- b�de i forhold til
spilprogressionen, samt den oplevelse de f�r.

N�r KI'en giver s� store muligheder for at p�virke spillet, er det
v�sentligt at \emph{game designeren} bruger en del tid p� at udt�nke
og finpudse den kunstige intelligens. I alle vores projekter var
gamedesigneren langt fra interreseret i den kunstige intelligens
p�virkning p� spillet: Der blev brugt mere tid p� detaljer i andre
dele af spillet. Det p� trods af at vi tydeligt kan se, at KI'en
spiller en stor rolle. Vi er sikre p�, at alle produktionerne kunne
v�re blevet bedre, hvis KI'en i vores spil havde f�et mere fokus i
designprocessen.

%%% Local Variables: 
%%% mode: latex
%%% TeX-master: "../report"
%%% End: 
