Vi har valgt at bruge heatmaps for bedre at kunne teste vores
hypotese. Det giver mange fordele for os at bruge heatmaps til at
analysere vores brugertest. Det vil v�re sv�rt for os at indsamle
korrekt data til test af vores hypotese, is�r da vi har b�rn som
m�lgruppe. B�rn kan have en tendens til at ville \emph{please} med at
give de resultater de tror man vil have.  Automatisk genererede
heatmaps er praktiske da vi ikke beh�ver v�re over b�rnene hele tiden
og se hvad de laver. De kan selv f� lov til at udforske banerne som de
ville hvis de skulle spille spillet derhjemme. Dermed har vi bedre
mulighed for at f� korrekt data ved at bruge heatmaps til at analysere
b�rnenes progression i spillene, i forhold til hvis vi skulle sidde og
se dem spille. Samtidig h�ber vi det kan give visuel data der allerede
er \emph{interpreted} til at give progressions information. Ved at se
med p� sk�rmen og optage b�rnene n�r de spiller spillene f�r vi deres
reaktions m�nstre, ved at optage heatmaps f�r vi deres bev�gelses- og
progressionsm�nstre. 