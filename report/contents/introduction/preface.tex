\chapter{Indledning}
\label{cha:indledning}

Vi har udviklet tre spil i forbindelse med et ophold i
'virksomhedspraktik' p� DADIU\citep{dadiu}. Vi indgik i
produktionshold p� cirka 15 mennesker fordelt over en r�kke
forskellige specialiteter.

Fra DADIU's side blev der udstukket begr�nsninger til spillene. Et af
dem var, at spillene skulle henvende sig til b�rn i 3-7-�rs alderen,
hvor det p� alle vores produktioner blev besluttet, at indskr�nke
intervallet til 6-7 �r. Et andet krav var, at der skulle v�re kunstig
intelligens, som en central del af \emph{gameplayet}.

Man kan derfor stille sig selv sp�rgsm�let, om tilstedev�relsen af
kunstig intelligens overhovedet g�r en forskel i spil rettet mod
b�rn. Bruger b�rnene samme m�ngde tid p� det samme spil, hvad enten
det indeholder en kunstig intelligens eller ej?

Sp�rgsm�let er interessant, da der bliver udviklet flere og flere
computerspil m�lrettet mod b�rn, og det derfor relevant at finde ud
af, om kunstig intelligens er en faktor, der skal overvejes, n�r et
computerspil designes. Sp�rgsm�let er endvidere interessant, idet der
ikke umiddelbart er nogen tidligere litteratur at finde omkring
kunstig intelligens' p�virkning af b�rns spiloplevelse.

Vi vil derfor, ved hj�lp af tre spil, hvor de kunstige intelligenser
spiller forskellige roller, unders�ge om den kunstige intelligens har
nogen indvirkning p� spiloplevelsen.

I rapporten vil vi begynde med at beskrive DADIU og herefter de tre
produktioner, vi hver i s�r var en del af. Vi vil herunder evaluere
ting, der har v�ret specifikke for vores produktioner. Efterf�lgende
vil vi evaluere DADIU samt udviklingsmilj�et Unity3D
(Unity)\citep{unity} generelt. Vi vil beskrive vores fors�g, hvad vi
m�ler, og hvorfor vi m�ler det. Vi vil beskrive vores testplan og
argumentere for, hvorfor den er udformet, som den er. Til slut vil vi
diskutere og evaluere vores resultater og spil p� baggrund af den
udf�rte test. Vi vil ogs� vurdere, om tilstedev�relsen og typen af en
kunstig intelligens, i et spil m�lrettet mod b�rn, er en faktor der
skal overvejes af \emph{game designere} i forbindelse med
spiloplevelsen.

%%% Local Variables: 
%%% mode: latex
%%% TeX-master: "../../report"
%%% End: 