
\chapter{Problemformulering}
\label{cha:problemformulering}

Vi vil i et praktikforl�b deltage i tre spilproduktioner som programm�rer. I
denne sammenh�ng vil vi unders�ge hvorvidt kunstig intelligens i computerspil
(KI) kan p�virke spillerens progression gennem spillet. Da spilproduktionernes
m�lgruppe er 6-7 �r, vil vores m�lgruppe v�re den samme. Vi skal derfor udvide
vores spil med funktionalitet der lader os m�le og registrere forskellige
variabler omkring spillerens progression.

Desuden vil vi beskrive produktionsforl�bet, det brugte udviklingsmilj� Unity
\footnote{\url{http://www.unity3d.com}}, samt de designvalg vi blev udsat for
undervejs. Desuden vil vi evaluere vores endelige spilprodukter. Vi vil ogs�
diskutere, hvorledes den kunstige intelligens eventuelt kan �ndres for at
underst�tte det af gamedesigneren udt�nkte prim�re m�l i spillet eller skabe
alternative m�l.

%%% Local Variables: 
%%% mode: latex
%%% TeX-master: "../report"
%%% End: 
