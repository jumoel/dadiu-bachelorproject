
\chapter{Problemformulering}
\label{cha:problemformulering}

Vi har i et praktikforl�b deltaget i tre spilproduktioner som programm�rer. I
denne sammenh�ng har vi unders�gt hvorvidt kunstig intelligens i computerspil
(KI) kan p�virke spillerens progression gennem spillet. Da spilproduktionernes
m�lgruppe var 6-7 �r, var vores m�lgruppe den samme. Vi skulle derfor udvide
vores spil med funktionalitet der lod os m�le og registrere forskellige
variabler omkring spillerens progression.

Desuden har vi beskrevet produktionsforl�bet, det brugte udviklingsmilj� Unity
\footnote{\url{http://www.unity3d.com}}, samt de designvalg vi blev udsat for
undervejs. Desuden har vi evalueret vores endelige spilprodukter. Vi har ogs�
diskuteret, hvorledes den kunstige intelligens eventuelt kunne �ndres for at
underst�tte det af gamedesigneren udt�nkte prim�re m�l i spillet eller skabe
alternative m�l.

%%% Local Variables: 
%%% mode: latex
%%% TeX-master: "../report"
%%% End: 
