\chapter{Form�l}
\label{cha:formal}

Man kan stille sig selv sp�rgsm�let om tilstedev�relsen af kunstig
intelligens overhovedet g�r en forskel i spil rettet mod b�rn. Bruger
b�rnene samme m�ngde tid p� det samme spil, hvad enten det indeholder
en kunstig intelligens eller ej?

Vi vil i denne rapport, ved hj�lp af tre spil hvor de kunstige
intelligenser spiller forskellige roller, unders�ge om den
kunstige intelligens har nogen indvirkning p� spiloplevelsen.

Idet m�lgruppen er b�rn vil vi holde os fra at fors�ge at m�le
subjektive vurderinger om 'sjovhed', evne til at underholde og
lignende kvalitetsangivelser. Det g�r vi, fordi det er en videnskab i
sig selv at f� relevant data ud af b�rn \citep{funtoolkit} \citep[side
4, f�rste afsnit]{childrenrespondents}. Derfor har vi valgt at
fokusere p� data, der beskriver spillerens progression gennem banerne,
og ud fra disse vurdere, om den kunstige intelligens giver grobund for
alternative spiloplevelser.

Endelig vil vi vurdere, om tilstedev�relsen og typen af en kunstig
intelligens i et spil m�lrettet mod b�rn, er en faktor der skal
overvejes af \emph{game designere}, i forbindelse med spiloplevelsen.

%%% Local Variables: 
%%% mode: latex
%%% TeX-master: "../../report"
%%% End: 

