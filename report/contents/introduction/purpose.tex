\chapter{Form�l}
\label{cha:formal}

Man kan stille sig selv sp�rgsm�let om tilstedev�relsen af kunstig
intelligens overhovedet g�r en forskel i spil rettet mod b�rn. Synes
de egentlig, det er lige s� sjovt bare at udforske eller spille p�
egen h�nd?

Vi vil i denne rapport, ved hj�lp af tre spil, hvor de kunstige
intelligenser spiller forskellige roller, unders�ge om den
kunstige intelligens har nogen indvirkning p� spiloplevelsen.

Idet der i m�lgruppen er tale om b�rn vil vi fraholde os fra at
fors�ge at m�le subjektive vurderinger om ``sjovhed'', evne til at
underholde og lignende kvalitetsangivelser. Det g�r vi, fordi det er
en videnskab i sig selv fx at udforme sp�rgeskemaer til b�rn [FIND
LITTERATUR OG INDS�T KILDE], hvorfra man rent faktisk kan f� relevant
data. Derfor har vi valgt at fokusere p� data der beskriver spillerens
progression gennem banerne og ud fra disse vurdere, om den kunstige
intelligens giver grobund for alternative spiloplevelser. Endelig vil
vi vurdere, om inkluderingen af kunstig intelligens i et spil
m�lrettet mod b�rn, er en faktor der skal overvejes af \emph{game
  designere} i forbindelse med alternative spiloplevelser, samt i
hvilken omfang man kan p�virke den spiloplevelse b�rnene f�r.

%%% Local Variables: 
%%% mode: latex
%%% TeX-master: "../../report"
%%% End: 

