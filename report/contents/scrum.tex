
\chapter{SCRUM}
\label{cha:scrum}

\emph{SCRUM} er en fleksibel m�de at udvikle software p�. I stedet for
de traditionelle udviklingsmetoder, som for eksempel vandfaldsmetoden,
erkender man i \emph{SCRUM}, at der opst�r uforudsigelige ting gennem
en udviklingsproces. Man har derfor lavet et system, som nemt kan
h�ndtere �ndringer. I \emph{SCRUM} arbejder holdet i 'sprints'. I
traditionel \emph{SCRUM} varer en sprint typisk to til fire uger. Ved
enden af hver sprint skal der v�re et potentielt f�rdigt produkt.

I \emph{SCRUM} har ejeren af produktet (ofte projektlederen) en
'product backlog'. Her kan ejeren l�gge s� mange krav i bunken, som
han lyster. Hvert krav skal have en prioritering, og kravene med
h�jest prioriteting er dem, der bliver lavet f�rst. Ved starten af
hver sprint estimerer holdet, hvor meget de kan n� i l�bet af
sprinten. De krav, de kan n� at opfylde flyttes over i en 'sprint
backlog'. Hver dag under sprinten holdes der et st�ende m�de, hvor alle p�
holdet fort�ller, hvordan tingene st�r til.

\emph{SCRUM} har en r�kke fordele i forhold til de traditionelle
metoder og is�r til mindre udviklingshold. F�rst og fremmest er der
stor fleksibilitet. Det er nemt at justere tingene, hvis noget tager
l�ngere tid end forventet, eller hvis der opst�r uventede
problemer. Det er ogs� nemmere at hj�lpe hinanden internt p� holdet,
da alle ved, hvad de andre laver, samt hvilke problemer der m�tte
v�re. Man har ogs� hele tiden et f�rdigt produkt der virker.

%%% Local Variables:
%%% mode: latex
%%% TeX-master: "../report"
%%% End: 