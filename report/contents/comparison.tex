
\chapter{Sammenligning af vores KI'er}
\label{cha:samm-af-vores-ki}


Selvom den kunstige intelligens i alle tre spil anvender simpel
pathfinding og derved minder meget om hinanden, udfylder de tre helt
forskellige roller.

I Catnapped optr�der KI'en i en rolle som direkte forhindring. Dens
eneste form�l er at g�re det sv�rere for spilleren at bev�ge sig rundt
i gangene, og spillerens tvinges til at time sine bev�gelser for at
pasere katte�jnene.

I imachination er KI'en flere ting. Den er til for at tvinge spilleren
til at bev�ge sig rundt i banen og ikke blot st� stille et sted og
skyde efter bygningerne. Derudover fungerer den som et vink til
spilleren om at bygningerne skal bek�mpes. Det signalerer de fordi de
bliver ved med at lave flere fjender �ndtil de er sl�et ihjel. �nsker
spilleren derfor at stoppe den uendelige str�m af modstandere m� han
bek�mpe bygningerne. Til sidst udfylder KI'en en ren tidsfordrivs
rolle. Nogle finder det sjovt blot at k�re rundt og skyde de sm�
modstandere med deres forskellige v�ben.

dreamleap - hj�lp I Sophies dream leap fungerer KI'en som en
guide. Dens rolle er at vise vejen gennem banen og skulle spilleren
v�re for langsom g�r den tilbage for at ``hent'' spilleren. Derved
arbejder den for at g�re spillet nemmere i mods�tning til de to andre
spil. KI'en fungerer ogs� som chekcpoint. Hver gang man indhenter den
gemmer spillet og man har mulighed for at live op samme sted, skulle
man falde ned og d�.

Hvad er det s� der er f�lles for vores tre kunstige intelligenser? I
alle spillene har de indflydelse p� spilleres progression gennem
spillet. Om det er hj�lp eller forhindring, eller om det skaber
alternative legemuligheder, p�virker de alle b�de hvor meget tid
spilleren bruger i spillet, men ogs� udfordringsniveauet. I de
efterf�lgende afsnit vil vi unders�ge i hvilket omfang vores tre KI'er
p�virker progressionen gennem spillene.



%%% Local Variables: 
%%% mode: latex
%%% TeX-master: "../report"
%%% End: 
