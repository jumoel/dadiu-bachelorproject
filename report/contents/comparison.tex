
\chapter{Sammenligning af vores KI'er}
\label{cha:samm-af-vores-ki}

Vi har alle tre haft pathfinding som vores prim�re KI. I \cat gik
det ud p� en simpel shortest path algoritme som kunne k�re i line�r
tid over antal �jne i labyrinten. Til \sop blev der lavet en A*
pathfinding algoritme til at styre broderen. Til \ima blev der fundet
en A* pathfinding algoritme p� nettet, denne styrede de sm�
modstandere der l�ber efter ens maskine.

Selvom den kunstige intelligens i alle tre spil anvender simpel
pathfinding og derved minder meget om hinanden, udfylder de tre helt
forskellige roller.

I \cat optr�der KI'en i en rolle som direkte forhindring. Dens
eneste form�l er at g�re det sv�rere for spilleren at bev�ge sig rundt
i gangene, og spillerens tvinges til at time sine bev�gelser for at
pasere katte�jnene.

I \ima er KI'en flere ting. Den er til for at tvinge spilleren til at
bev�ge sig rundt i banen og ikke blot st� stille et sted og skyde
efter bygningerne. Derudover fungerer den som et vink til spilleren om
at bygningerne skal bek�mpes. Det skyldes at bygningerne bliver ved
med at lave fjender, indtil de er blevet sl�et ihjel. �nsker spilleren
derfor at stoppe den uendelige str�m af modstandere m� han bek�mpe
bygningerne. Til sidst udfylder KI'en en ren tidsfordrivsrolle. Nogle
finder det sjovt blot at k�re rundt og skyde de sm� modstandere med
deres forskellige v�ben.

I \sop fungerer KI'en som en guide. Dens rolle er at vise vejen gennem
banen. Spillet bliver ogs� gemt, n�r man indhenter KI'en p� bestemte
steder i banen. Det er ved ens seneste gemmepunkt at man bliver
genoplivet, hvis man d�r. Dermed arbejder den for at g�re spillet
nemmere, i mods�tning til de to andre spil.

Hvad er det s� der er f�lles for vores tre kunstige intelligenser? I
alle spillene har de indflydelse p� spilleres progression gennem
spillet. Om det er hj�lp eller forhindring, eller om det skaber
alternative legemuligheder, p�virker de alle b�de hvor meget tid
spilleren bruger i spillet, men ogs� udfordringsniveauet.

%%% Local Variables: 
%%% mode: latex
%%% TeX-master: "../report"
%%% End: 
