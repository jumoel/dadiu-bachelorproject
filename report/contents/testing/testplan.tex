
\section{Testplan}
\label{sec:testplan}

For at f� de bedst mulige resultater ud af vores test, er vi n�dt til
at l�gge lidt tanker bag den m�de og r�kkef�lge vi tester vores spil
p�.

Vi tester to udgaver af hvert spil (�n uden kunstig intelligens
og �n med) og vi vil gerne se, om der er en forskel i m�den, spillet
spilles p�. Derfor skal vi b�de have b�rn der spiller udgaven
\emph{med} kunstig intelligens f�rst og b�rn der spiller udgaven
\emph{uden} kunsting intelligens f�rst.

Fordi styringen i de tre spil er i samme kategori kan det ogs� have
betydning, om spillet de pr�ver er det f�rste spil eller et
efterf�lgende spil. Derfor skal alle spil spilles b�de som nummer �t
og som nummer to eller tre.

Da der er tre spil og hvert spil er i to versioner, skal vi minimum
bruge seks b�rn. Seks b�rn giver os mulighed for at teste hvert spil
som ``f�rste spil'' i begge udgaver. Desuden vil vi ogs� kunne teste
alle udgaver af alle spil �n gang som ``andet spil'' og �n gang som
``tredje spil''.  Vi h�ber at seks resultater for hvert spil vil give
os nok data til at finde en sammenh�ng.

% For at nogle b�rn ikke udelukkende har spillet KI-udgaverne af
% spillene f�rst, varierer vi hvilken type, b�rnene starter med at
% spille.

[TODO: SKAL VI VARIERE R�KKEF�LGEN, SOM FX I-C-S-I-C-S ELLER SKAL DET
V�RE I-I-C-C-S-S?]

Vi har p� baggrund af ovenst�ende, valgt at producere f�lgende
testplan:

\begin{tabular}[h]{llll}
  \textbf{Barn & Spil 1 & Spil 2 & Spil 3} 
\end{tabular}

[TODO: R�KKEF�LGE, JEG KAN IKKE LIGE OVERSKUE DET :)]

%%% Local Variables: 
%%% mode: latex
%%% TeX-master: "../../report"
%%% End: 
