
\section{Testresultater}
\label{sec:testresultater}

\begin{table}
\begin{longtable}{|c|cccccc|} \hline 

\textbf{Barn} & \textbf{Spil 1} & \textbf{Spil 2} & \textbf{Spil 3} & \textbf{Spil 4} &
\textbf{Spil 5} & \textbf{Spil 6}\\
\hline
1D & i- & cx & s- & ix & c-\# & s\#\\
2P & i & s-x & c-\# & i-x & s\# & c\#\\
3D & c- & c & i & i- & s & s-x\\
4P & s & s-x & ix & i-x & cx & c-x \\
5D & s & s- & i- & i & c & c-\\
6D & i- & i & cx & c-x & s- & s \\
7D & s- & s & i- & i\# & c\# & c-\#\\
8D & c & c-& s- & s & i- & i \\
9D & s & s- & i- & i & c & c-\\
10D & s & s- & c\# & c-\# & i\# & i-\#\\\hline
\multicolumn{7}{|p{9cm}|}{\emph{i}, \emph{s} og \emph{c} angiver hhv. \ima,
  \sop og \cat.}\\
\multicolumn{7}{|p{9cm}|}{\emph{-} betyder at spillet er spillet uden KI}\\
\multicolumn{7}{|p{9cm}|}{Et \emph{x} angiver at barnet gav op undervejs i
  spillet. \emph{\#} angiver at barnet ikke har spillet.}\\
\multicolumn{7}{|p{9cm}|}{\emph{\#} angiver at barnet ikke har spillet.}\\\hline

\end{longtable}
\caption{Testresultat TODO FIX}
\label{tbl:testresultat}
\end{table}
%%% Local Variables: 
%%% mode: latex 
%%% TeX-master: "../../report"
%%% End:
