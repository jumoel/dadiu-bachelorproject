
\section{Forventninger ved brugertest}
\label{sec:brugertestforventninger}

Som resultatet af vores brugertest, forventer vi at kunne se klare
�ndringer p� de heatmaps fra vores spil hhv. med og uden KI.

I \cat forventer vi at spilleren kommer til at bev�ge sig
hurtigere rundt i labyrinten. Om det vil f�re til at de hurtigere
kommer ud af den, er ikke til at sige. B�rnenes valg bliver muligvis
mindre velovervejede, da der ikke l�ngere er fare ved at bev�ge sig rundt.

I \ima forventer vi at banerne bliver klaret hurtigere. Der er ikke
l�ngere et krav om at bev�ge sig for at holde sig i live. Det er
muligt blot at skyde bygningerne p� afstand uden den store
udfordring. Spilleren vil hovedsageligt opholde sig helt t�t p�
bygningerne da de ikke bliver tvunget til at k�re rundt i resten af
banen. Der er heller ikke l�ngere mulighed for at skyde ting uden at
komme t�ttere p� at gennemf�re banen. Desuden er der ikke nogen
mulighed for at b�rnene fokuserer p� blot at dr�be de sm� fjender.

I \sop forventer vi at det tager l�ngere tid at komme gennem
banerne. Der vil v�re mere udforskning da der ikke l�ngere er en guide
til at pege dig i en bestemt retning. Vi forventer ogs� at spilleren
vil d� mere da de formentlig vil fors�ge at lave sv�rere
spring. Vejene spillerne vil tage, forventer vi vil afvige meget.

Vi forventer at se en betydelig forskel p� de b�rn der spiller med KI
f�rst og dem der spiller uden KI f�rst. Fx er \sop en del nemmere at
spille uden KI hvis f�rst du har spillet det med, da man kender vejen
rundt i banen.

%%% Local Variables: 
%%% mode: latex
%%% TeX-master: "../../report"
%%% End: 