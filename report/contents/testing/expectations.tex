
\section{Forventninger ved brugertest}
\label{sec:brugertestforventninger}

Vi forventer at kunne se klare �ndringer i vores heatmaps med og uden
KI i vores spil.

I Catnapped forventer vi at spilleren kommer til at bev�ge sig
hurtigere rundt i labyrinten. Om det vil f�re til at de hurtigere
kommer ud af den er ikke til at sige. B�rnenes valg bliver muligvis
mindre velovervejede, da der ikke er fare ved at bev�ge sig rundt
mere.

I Imachination forventer vi at banerne bliver klaret hurtigere. Der er
ikke l�ngere et krav om at bev�ge sig for at holde sig i live. Det er
muligt blot at skyde bygningerne p� afstand uden den store
udfordring. Spilleren vil hovedsageligt opholde sig helt t�t p�
bygningerne da de ikke bliver tvunget til at k�re rundt i resten af
banen. Der er heller ikke l�ngere mulighed for at skyde ting uden at
komme t�ttere p� at gennemf�re banen.

I Sophies dreamleap forventer vi at det tager l�ngere tid at komme
gennem banerne. Der vil v�re mere udforskning da der ikke l�ngere er
en guide til at pege dig i den rigtige retning. Vi forventer ogs� at
spilleren vil d� mere da de formentlig vil fors�ge at lave sv�rere
spring.

Vi forventer at se en betydelig forskel p� de b�rn der spiller med KI
f�rst og dem der spiller uden KI f�rst. Fx er Sophies dreamleap en del
nemmere at spille uden KI hvis f�rst du har spillet det med, og kender
vejen rundt i banen.

%%% Local Variables: 
%%% mode: latex
%%% TeX-master: "../../report"
%%% End: 