
\section{Refleksion over produktionerne}
\label{sec:refl-over-prod}

Efter at have v�ret ude at teste i fritidshjemmet er vi blevet meget
kritiske over for de test, der blev lavet p� vores produktioner. I alle
tilf�lde var det gamedesigner, instrukt�r og projektleder, der var ude
at teste spillene med b�rnene. Enten har det feedback de har givet
videre til resten af gruppen ikke v�re korrekt, eller ogs� har testen
v�ret meget ringe udf�rt. Der viste sig tydelige problemer under vores
forholdsvis simple test af spillene. P� den �ne dag vi har testet
spillene, er det gennemg�ende resultatat, spillene har v�ret for
sv�re. Leveldesignet p� banerne har v�ret langt fra optimalt og
b�rnene har haft problemer med styringen. Flere af b�rnene har givet
op p� spillene, fordi de ikke har f�et nok feedback fra spillet og er
g�et i st�. Mange af disse problemer kunne have v�ret afhjulpet hvis
holdet havde investeret noget tid i at l�se dem og det havde hjulpet
meget p� spiloplevelsen.

Vi har en mistanke om at b�rnene blevet hjulpet undervejs i
testene. Hvis det er tilf�ldet at b�rnene ikke har spillet 100\% selv,
kan man i realiteten ikke rigtig resultaterne til noget. Vi har derfor
konstateret at der til n�ste DADIU-produktion skal v�re programm�rer
med til testene, da de studerende fra de andre uddannelser
tilsyneladende ikke f�r tilstr�kkelig tr�ning eller undervisning i,
hvordan man tester software.

%%% Local Variables: 
%%% mode: latex
%%% TeX-master: "../../report"
%%% End: 
