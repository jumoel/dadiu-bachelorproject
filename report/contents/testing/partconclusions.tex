
\section{Delkonklusioner}
\label{sec:delkonklusioner}

\subsection{\ima}
Hvis man kigger p� heatmapsne i afsnit \ref{sec:ima-heatmaps} p� side
\pageref{sec:ima-heatmaps} kan man se en gennemg�ende tendens til mere
bev�gelse n�r b�rnene spiller med KI. Flere af b�rnene udtrykte
begejstring over at skyde de sm� fjender og f� lov til at se dem flyve
ud af sk�rmen. Derfor er der g�et en del tid med b�de at undvige
fjenderne for ikke at d�, men ogs� med at skyde dem for morskabens
skyld. Man kan med stor sikkerhed antage at den af versionerne b�rnene
har spillet f�rst tager, relativt til den anden, l�ngere tid idet de
skal l�re spillet. N�r de derimod kommer til at anden udgave har de
allerede alle de fundementale ting om spillet p� plads.  Man kan se af
tabel \ref{antage at alle dem der} p� side \pageref{antage at alle dem
  der} at tiderne med KI er h�jere end tiderne uden. Og i de f�
tilf�lde hvor tiden uden KI enten er h�jere eller omtrent lige s� h�j
som tiden med KI har spilleren spille uden KI f�rst og derved har
indl�ringstiden med heri. i P4's tilf�lde opgav hun spilled med KI da
hun d�de og derfor er hendes tid med KI lavere end den uden.

At KI'en fungerer som et hint til spilleren om at bygningerne skal
bek�mpes har vi et fremragende eksempel p�. Resultaterne fra 4P viser
at hun ikke kunne finde ud af hvad m�let var med spillet og hvordan
hun skulle komme videre. Det lykkedes hende i testen at ``snyde'' sig
igennem tutorialen og derved ikke forstod at hun skulle smadre
bygningerne for at komme videre i banen. Derfor er hendes heatmap i
versionen uden KI, som hun spillede f�rst, med meget mere bev�gelse
end hun beh�vede fordi hun blot k�rte rundt uden at kunne finde ud af
hvad hun skulle. 

\subsection{\sop}

\subsection{\cat}



%%% Local Variables: 
%%% mode: latex 
%%% TeX-master: "../../report"
%%% End:
