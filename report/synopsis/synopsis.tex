\documentclass[10pt,a4paper,danish]{article}

\usepackage[utf8x]{inputenc}
\usepackage[T1]{fontenc}
\usepackage[danish]{babel}
\usepackage{amsmath}
\usepackage{amsfonts}
\usepackage{amssymb}
\usepackage{fancyhdr}
\usepackage{graphicx}
\usepackage{lastpage}
\usepackage{listings}
\usepackage{charter}
\usepackage{url}

\newcommand{\shorttitle}{Kunstig intelligens og spillerens progression i computerspil}
\newcommand{\thetitle}{Synopsis for bachelorprojekt, blok 4 2009/2010}

\title{Synopsis til \thetitle}

\date{21. maj 2010}
\newcommand{\theauthors}{Julian Møller, Steen Nordsmark Pedersen \& Klaes Bo Rasmussen}
\author{\theauthors}
\setlength{\headheight}{15pt}

\pagestyle{fancy}
	\lhead{\thetitle}
	\chead{}
	\rhead{}
	\lfoot{}
	\cfoot{Side \thepage \ af \pageref{LastPage}}
	\rfoot{}

\begin{document}

\begin{tabbing}
\textbf{Forfattere} \= \theauthors \\
\textbf{Titel} \> \thetitle
\end{tabbing}

\section{Problemformulering}
\label{sec:Problemformulering}
Vi vil i et praktikforløb deltage i tre spilproduktioner som programm�rer. I
denne sammenh�ng vil vi unders�ge hvorvidt kunstig intelligens i computerspil
(KI) kan p�virke spillerens progression gennem spillet. Da spilproduktionernes
m�lgruppe er 6-7 �r, vil vores m�lgruppe v�re den samme. Vi skal derfor udvide
vores spil med funktionalitet der lader os m�le og registrere forskellige
variabler omkring spillerens progression.

Desuden vil vi beskrive produktionsforl�bet, det brugte udviklingsmilj� Unity
\footnote{\url{http://www.unity3d.com}}, samt de designvalg vi blev udsat for
undervejs. Desuden vil vi evaluere vores endelige spilprodukter. Vi vil ogs�
diskutere, hvorledes den kunstige intelligens eventuelt kan �ndres for at
underst�tte det af gamedesigneren udt�nkte prim�re m�l i spillet eller skabe
alternative m�l.

\section{Afgr�nsning}
\label{sec:Afgraensning}
Vores praktikforl�b, rapportskrivning og udvikling vil udelukkende finde sted
i blok 4. Endvidere er der fra DADIUs side udstukket begr�nsninger for spilm�l-
gruppen. Dette p�virker vores muligheder for at unders�ge effekten af KI, da vi har
med b�rn at g�re. Fx bliver sp�rgeskemaer vanskeligere at f� relevant information
ud fra. Derfor vil vores opgave ikke besk�ftige sig med f�lelsesm�ssige eller
kvalitetsbaserede opfattelser, men blot progression igennem spillene i form af fx
tid og placering.

Da vi er p� tre forskellige spilproduktioner er det ikke sikkert at alle tre
spil vil give lige god mulighed for at unders�ge vores �nskede hypotese.

\section{Begrundelse}
\label{sec:Begrundelse}
DADIU-produktioner giver et virkelighedstro milj� at lave spil i, hvilket kan
give os mulighed for at unders�ge vores hypotese under realistiske forhold,
idet spillene ikke er lavet for at underst�tte vores hypotese, men for at v�re
spil.

\section{Evaluering}
\label{sec:Evaluering}
Vi vil betegne vores projekt som v�rende succesfuldt, hvis:

\begin{itemize}
\item vi f�r udviklet funktionalitet, der tillader os at m�le og registrere
forskellige variable omkring spillerens progression.
\item vi f�r beskrevet hypotesen og hvorledes vi tror den kan eftervises
\item vi finder relevant teori og f�r perspektiveret vores opgave i forhold
til den
\item vi f�r unders�gt vores hypotese vha. personer fra m�lgruppen
\end{itemize}

\end{document}
