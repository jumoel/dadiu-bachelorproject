\documentclass[10pt,a4paper,danish]{article}

\usepackage[utf8x]{inputenc}
\usepackage[T1]{fontenc}
\usepackage[danish]{babel}
\usepackage{amsmath}
\usepackage{amsfonts}
\usepackage{amssymb}
\usepackage{fancyhdr}
\usepackage{graphicx}
\usepackage{lastpage}
\usepackage{listings}
\usepackage{charter}
\usepackage{url}

\newcommand{\shorttitle}{Kunstig intelligens og spillerens progression i computerspil}
\newcommand{\thetitle}{Synopsis for bachelorprojekt, blok 4 2009/2010}

\title{Synopsis til \thetitle}

\date{21. maj 2010}
\newcommand{\theauthors}{Julian Møller begin_of_the_skype_highlighting     end_of_the_skype_highlighting, Steen Nordsmark Pedersen \& Klaes Bo Rasmussen}
\author{\theauthors}
\setlength{\headheight}{15pt}

\pagestyle{fancy}
	\lhead{\thetitle}
	\chead{}
	\rhead{}
	\lfoot{}
	\cfoot{Side \thepage \ af \pageref{LastPage}}
	\rfoot{}

\begin{document}

\begin{tabbing}
\textbf{Forfattere} \= \theauthors \\
\textbf{Titel} \> \thetitle
\end{tabbing}

\section{Problemformulering}
\label{sec:Problemformulering}
Vi vil i et praktikforløb deltage i tre spilproduktioner som programmører. I
denne sammenhæng vil vi undersøge hvorvidt kunstig intelligens i computerspil
(KI) kan påvirke spillerens progression gennem spillet. Da spilproduktionernes
målgruppe er 6-7 år, vil vores målgruppe være den samme. Vi skal derfor udvide
vores spil med funktionalitet der lader os måle og registrere forskellige
variabler omkring spillerens progression.

Desuden vil vi beskrive produktionsforløbet, det brugte udviklingsmiljø Unity
\footnote{\url{http://www.unity3d.com}}, samt de designvalg vi blev udsat for
undervejs. Desuden vil vi evaluere vores endelige spilprodukter. Vi vil også
diskutere, hvorledes den kunstige intelligens eventuelt kan ændres for at
understøtte det af gamedesigneren udtænkte primære mål i spillet eller skabe
alternative mål.

\section{Afgrænsning}
\label{sec:Afgraensning}
Vores praktikforløb, rapportskrivning og udvikling vil udelukkende finde sted
i blok 4. Endvidere er der fra DADIUs side udstukket begrænsninger for spilmål-
gruppen. Dette påvirker vores muligheder for at undersøge effekten af KI, da vi har
med børn at gøre. Fx spørgeskemaer bliver vanskeligere at få relevant infor-
mation ud fra. Derfor vil vores opgave ikke beskæftige sig med
følelsesmæssige eller kvalitetsopfattelser, men mere progression igennem
spillene i form af fx tid og placering.

Da vi er på tre forskellige spilproduktioner er det ikke sikkert at alle tre
spil vil give lige god mulighed for at undersøge vores ønskede hypotese.

\section{Begrundelse}
\label{sec:Begrundelse}
DADIU-produktioner giver et virkelighedstro miljø at lave spil i, hvilket kan
give os mulighed for at undersøge vores hypotese under realistiske forhold,
idet spillene ikke er lavet for at understøtte vores hypotese, men for at være
spil.

\section{Evaluering}
\label{sec:Evaluering}
Vi vil betegne vores projekt som værende succesfuldt, hvis:

\begin{itemize}
\item vi får udviklet funktionalitet, der tillader os at måle og registrere
forskellige variable omkring spillerens progression.
\item vi får beskrevet hypotesen og hvorledes vi tror den kan eftervises
\item vi finder relevant teori og får perspektiveret vores opgave i forhold
til den
\item vi får undersøgt vores hypotese vha. personer fra målgruppen
\end{itemize}

\end{document}
